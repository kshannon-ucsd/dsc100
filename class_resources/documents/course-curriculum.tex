\documentclass[11pt, letterpaper]{article}
\usepackage{../preamble}

% ----------------------- Specific Set up for this note -----------------------
\usepackage{tocloft}
\renewcommand{\cftsecleader}{\cftdotfill{\cftdotsep}} % for leader dots
\setlength{\cftsecindent}{0pt} % remove indentation for section titles
\setlength{\cftsecnumwidth}{0pt} % remove space for section number

\let\oldsection\section
\renewcommand\section{\clearpage\oldsection}

% ----------------------- Specific Set up for this note -----------------------


\title{WIP - DSC 100 Course Notes}
\subtitle{UC San Diego}
\author{Kyle Shannon}
\date{July 1, 2023}


\begin{document}
	\thispagestyle{empty}
	\maketitle
	\tableofcontents
	
	\newpage
	
	\section*{Lecture 1: Introduction and Overview}
	\addcontentsline{toc}{section}{Lecture 1: Introduction and Overview}
	In the first lecture, we will cover the basics and the importance of databases in Society, Data Science, AI, and ML. Students will understand the role and significance of databases in various applications, and learn about some of the history of databases. We will cover course expectations and onboarding.
	\begin{itemize}
		\item Course introduction \& onboarding
		\item Importance of Databases in analytical and nonanalytical roles
		\item Data, Databases, and DBMS
		\item Brief on Databases in DS, AI, and ML Applications with some examples
	\end{itemize}
	Relevant Resource: \href{https://example.com}{TBD}
	
	\section*{Lecture 2: The Relational Data Model}
	\addcontentsline{toc}{section}{Lecture 2: The Relational Data Model}
	The second lecture focuses on the key differences between a relational data model and other data models students are likely to have used (e.g. csv, excel, and data frames). Students will learn the basic concepts of data models and the fundamentals of relational databases through n interactive demo. The will also learn about SQLite and Database Management Systems.
	\begin{itemize}
		\item Basic concepts of data models
		\item Introduction \& Terminology to Relational Databases
		\item  Database Design for DS Projects 
		\item SQLite Demo
	\end{itemize}
	Relevant Resource: \href{https://example.com}{TBD}
	
	\section*{Lecture: Introduction to SQL}
	The third lecture introduces SQL, a vital language for managing and manipulating databases. Students will leanr about basic commands, and how SQL is used in practice. There will be several in-class demos.
	\begin{itemize}
		\item Importance of SQL
		\item Basic SQL commands (SELECT, FROM, WHERE)
		\item - DDL (CREATE, ALTER, DROP)
		\item - DML (INSERT, UPDATE, DELETE)
		\item Writing SQL in analytical uses vs. end user applications (e.g. a bank interface)
	\end{itemize}
	Relevant Resource: \href{https://example.com}{TBD}
	
	\section*{Lecture: SQL Joins I}
	In this lecture, students will gain understanding of how to use SQL joins and aggregates. The knowledge of these techniques will be crucial in retrieving data for complex analysis tasks.
	\begin{itemize}
		\item SQL JOIN operations {Inner, Outer, \& Self}
	\end{itemize}
	Relevant Resource: \href{https://example.com}{TBD}
	
	\section*{Lecture: SQL Joins  II}
	In this lecture, We will continue to focus on joins, but use them in slightly more complex manners.
	\begin{itemize}
		\item SQL JOIN operations {Inner, Outer, \& Self}
		\item Using SQL joins for data analysis
	\end{itemize}
	Relevant Resource: \href{https://example.com}{TBD}
	
	\section*{Lecture: Aggregates \& Groupings}
	In this lecture, We will continue to focus on aggregates, \& groupings to solve more complex problems.
	\begin{itemize}
		\item SQL aggregate functions {SUM, COUNT, AVG, MAX, MIN}
		\item Using SQL joins, aggregates, \& groupings for data analysis
	\end{itemize}
	Relevant Resource: \href{https://example.com}{TBD}
		
	\section*{Lecture: Subqueries and Advanced Queries I}
	This lecture will delve deeper into SQL by explaining subqueries, which are instrumental for complex data manipulations and analysis.
	\begin{itemize}
		\item Introduction to SQL subqueries
		\item Using subqueries and complex queries for advanced data analysis
	\end{itemize}
	Relevant Resource: \href{https://example.com}{TBD}
	
	\section*{Lecture: Subqueries and Advanced Queries II}
	This lecture will delve deeper into SQL by explaining subqueries, which are instrumental for complex data manipulations and analysis.
	\begin{itemize}
		\item Introduction to SQL subqueries
		\item Using subqueries and complex queries for advanced data analysis
	\end{itemize}
	Relevant Resource: \href{https://example.com}{TBD}
	
	\section*{Lecture: Set Operations \& NULL}
	This lecture will delve deeper into SQL by explaining subqueries, which are instrumental for complex data manipulations and analysis.
	\begin{itemize}
		\item Set operations (UNION, INTERSECT, EXCEPT)
		\item Null values and their treatment in SQL
		\item Case Study: SQL for Data Cleaning in DS
	\end{itemize}
	Relevant Resource: \href{https://example.com}{TBD}
	
	\section*{Lecture: Relational Algebra}
	This lecture will delve deeper into SQL by explaining subqueries, which are instrumental for complex data manipulations and analysis.
	\begin{itemize}
		\item Relational Algebra's connection to SQL and DataFrames
	\end{itemize}
	Relevant Resource: \href{https://example.com}{TBD}
	
	\section*{Midterm}
	Midterm exam in person and will cover material up to Lecture: Subqueries and Advanced Queries
	
	\section*{Lecture: SQL for Data Cleaning \& Data Analysis}
	In this lecture, students will learn the importance of data cleaning in data science and how SQL can be used for this purpose.
	\begin{itemize}
		\item Importance of data cleaning in data science
		\item Using SQL for data cleaning
		\item Advanced SQL queries for data analysis
		\item Window functions (ROW\_NUMBER, RANK, DENSE\_RANK, etc.)
		\item PIVOT, UNPIVOT
	\end{itemize}
	Relevant Resource: \href{https://example.com}{TBD}
	
	\section*{Lecture: Storage, Indexing, \& Size Estimation}
	This lecture introduces the concept of indexing, which is crucial for improving database query performance.
	\begin{itemize}
		\item Introduction to indexing
		\item How indexing works (B-tree, hash index)
		\item How data is stored on disk \& accessed
		\item Size estimation techniques and sharding
		\item Impact of indexing on data retrieval in data science
	\end{itemize}
	Relevant Resource: \href{https://example.com}{TBD}
	
	\section*{Lecture: Database Design Principles \& ER Diagrams I}
	In this lecture, students will learn the principles of database design and the concept of E-R diagrams, which are crucial for effectively representing data relationships.
	\begin{itemize}
		\item Introduction to database design principles
		\item Understanding E-R diagrams
		\item Use of E-R diagrams in database design for data science projects
		\item Understand schema design
	\end{itemize}
	Relevant Resource: \href{https://example.com}{TBD}
	
	\section*{Lecture: Database Design Principles \& ER Diagrams II}
	In this lecture, students will learn the principles of database design and the concept of E-R diagrams, which are crucial for effectively representing data relationships.
	\begin{itemize}
		\item Introduction to database design principles
		\item Understanding E-R diagrams
		\item Use of E-R diagrams in database design for data science projects
		\item Understand schema design
		\item Views, Stored Procedures, Triggers
		\item Transactions
	\end{itemize}
	Relevant Resource: \href{https://example.com}{TBD}
	
	\section*{Lecture: Relational Data Modelling and Introduction to Normalization}
	Students will learn about relational data modeling and be introduced to normalization, a process to eliminate redundancy and dependency.
	\begin{itemize}
		\item Introduction to relational data modeling
		\item Introduction to normalization
		\item Use of normalization in data science
	\end{itemize}
	Relevant Resource: \href{https://example.com}{TBD}
	
	\section*{Lecture: Advanced Normalization}
	This lecture dives deeper into the concept of normalization, covering advanced normalization techniques. Students will learn how to optimize database designs to ensure data integrity and efficiency.
	\begin{itemize}
		\item Advanced normalization techniques (3NF, BCNF)
		\item Normalization in database design for data science projects
	\end{itemize}
	Relevant Resource: \href{https://example.com}{TBD}
	
	\section*{Lecture: Database Systems in AI and ML Applications}
	Students will learn how databases are used in AI and ML applications, with focus on data storage, retrieval and manipulation for model training and prediction.
	\begin{itemize}
		\item Role of databases in AI and ML
		\item Importance of database design for AI and ML applications
		\item Case studies of databases in AI and ML
	\end{itemize}
	Relevant Resource: \href{https://example.com}{TBD}
	
	\section*{Lecture: Concurrency \& ACID/BASE Properties of SQL \& NOSQL}
	This lecture dives deeper into the concept of normalization, covering advanced normalization techniques. Students will learn how to optimize database designs to ensure data integrity and efficiency.
	\begin{itemize}
		\item ACID vs BASE
		\item Concurrency
		\item Types of NoSQL databases (Document, Key-value, Column, Graph)
	\end{itemize}
	Relevant Resource: \href{https://example.com}{TBD}
	
	\section*{Lecture: Security, 'The Cloud', \& Warehousing}
	The lecture discusses the importance of database security and how to implement security measures to protect sensitive data. And students will also learn about cloud-based database systems, which are becoming increasingly popular in the data science industry.
	\begin{itemize}
		\item Importance of database security
		\item SQL for database security (GRANT, REVOKE)
		\item Introduction to cloud databases
		\item Advantages and challenges of cloud databases
		\item Cloud databases in data science
		\item Data Ingestion and Preparation in DS
	\end{itemize}
	Relevant Resource: \href{https://example.com}{TBD}
	
	
	\section*{Final Exam or Project}
	The final exam will cover materials from all the lectures.

\end{document}


Absolutely, here is a revised curriculum incorporating more Data Science and Machine Learning elements:

Week 1:
1. Introduction to Databases: History, Importance, and Applications
2. Database Concepts: Data Models, Schemas, and Instances

Week 2:
1. Introduction to RDBMS
2. RDBMS vs. other types of DBMS: Use Cases in Data Science

Week 3:
1. SQL Basics: Data Definition Language (DDL)
2. SQL Basics: Data Manipulation Language (DML) & Data Query Language (DQL)

Week 4:
1. Advanced SQL: Joins, Subqueries, Set Operations, and Aggregations
2. SQL for Data Science: Exploratory Data Analysis using SQL

Week 5:
1. Introduction to Data Warehousing: Importance and Use Cases
2. OLTP vs OLAP: Differences and Implications for Data Science

Week 6:
1. In-Class Midterm Examination (Covers Week 1 - Week 5)
2. Post-Exam Review and Introduction to Dimensional Modeling

Week 7:
1. Dimensional Modeling: Star Schema, Snowflake Schema, and Fact Constellation
2. SQL for Data Science: Analytical Functions and Window Functions

Week 8:
1. Data Normalization and Denormalization: Trade-offs and Use Cases in Data Science
2. Introduction to Indexing: Importance and Use Cases in Data Science

Week 9:
1. Database Connectivity with Python/R: Pulling Data into Pandas/R Dataframes
2. Introduction to Big Data Systems: Hadoop, Hive, and Spark

Week 10:
1. Databases in Machine Learning: Feature Storage, Model Storage, and Serving
2. Review, Q&A, and Wrap-Up

The same format applies - about an hour of theoretical explanations, followed by hands-on exercises or demos for about 20 minutes. Regular assignments related to the week's topics would help students to cement their understanding. Again, this is a guide and should be adapted according to the needs and skills of your students.

----

Sure, here's a suggested curriculum for your 10-week database management class. Each topic will be spread across two classes in a week.

Week 1:
1. Introduction to Databases: History, Importance, and Applications
2. Database Concepts: Data Models, Schemas, and Instances

Week 2:
1. Introduction to Relational Database Management Systems (RDBMS)
2. RDBMS vs. other types of DBMS, and Use Cases

Week 3:
1. Introduction to SQL: Importance and Applications
2. SQL Basics: Data Definition Language (DDL)

Week 4:
1. SQL Basics Continued: Data Manipulation Language (DML)
2. SQL Basics Continued: Data Query Language (DQL)

Week 5:
1. Advanced SQL: Joins, Subqueries, and Set Operations
2. Review and In-Class Practice Problems

Week 6:
1. In-Class Midterm Examination (Covers Week 1 - Week 5)
2. Post-Exam Review and Introduction to Data Normalization

Week 7:
1. Data Normalization: Importance, Normal Forms, and Use Cases
2. Data Normalization Continued: Practice Problems

Week 8:
1. Database Design: ER Models, ER to Relational Conversion
2. Database Design Continued: Practice Problems

Week 9:
1. Introduction to Transactions: ACID Properties, Concurrency Control
2. Introduction to Database Security: Data Integrity, Authentication, and Authorization

Week 10:
1. Database Administration: Backups, Recovery, Performance Tuning
2. Final Review and In-Class Practice Problems

Each class can be divided into two parts: theoretical explanations (about 1 hour) and hands-on exercises or demos (about 20 minutes). For the hands-on part, encourage students to solve practical problems using SQL and design databases for given scenarios.

You can assign weekly homework related to the topics covered in the week for reinforcement and continuous assessment.

Remember, this is just a suggestion. Depending on the skill level and interests of your class, you may need to adjust this plan to ensure the best learning outcome.
